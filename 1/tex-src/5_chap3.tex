\chapter{Выводы}
\label{ch:tab}

Итак, в данной работе было рассмотрено несколько вариаций реализации метода градиентного спуска для нахождения наиболее оптимального(минимального) значения функции нескольких переменных. Как оказалось, у различных методов имеются свои сильные и слабые стороны. Кроме того, различные модификации алгоритма в некоторых случаях ускоряют процесс поиска, а других, наоборот.

    Как оказалось, на сходимость методов градиентного спуска сильно влияет длина шага. Мы рассмотрели метод дихотомии и критерий Вольфе, с помощью которых мы можем подобрать более удачную длину шага. Стандартный градиентный спуск плохо справляется на некоторых функциях. В большинстве случаев градиентный спуск с использованием дихотомии работает за такое же количество итераций, но есть функции, на которых это количество уменьшается в разы. Градиентный спуск же с использованием критерия Вольфе показал себя как достаточно универсальный метод, работающий в большинстве случаев быстрее стандартного градиентного спуска.

    Нами так же было рассмотрено влияние различных параметров на скорость сходимости методов градиентного спуска. Мы сделали следующие выводы:
    \begin{itemize}
        \item С ростом числа обусловленности алгоритмы градиентного спуска сходятся быстрее.
        \item С ростом размерности функции алгоритмы градиентного спуска требуют больше итераций для завершения.
    \end{itemize}

\endinput