\chapter*{Введение}
\addcontentsline{toc}{chapter}{Введение}
\label{ch:intro}
    \textbf{Постановка задачи:}

    \begin{enumerate}
        \item Реализация градиентного спуска с постоянным шагом (learning rate).
        \item Реализация метода одномерного поиска - метод дихотомии и градиентный спуск на его основе.
        \item Анализ траектории градиентного спуска на примере различных квадратичных
        функций.
        \item Для каждой функции:
        \begin{enumerate}[label=(\alph*)]
            \item исследовать сходимость градиентного спуска с постоянным шагом 
            и сравнить
            полученные результаты для выбранных функций;
            \item сравнить эффективность градиентного спуска с использованием одномерного поиска с точки зрения количества вычислений минимизируемой
            функции и ее градиентов;
            \item
            исследовать работу методов в зависимости от выбора начальной точки;
            \item исследовать влияние нормализации (scaling) на сходимость на примере масштабирования осей плохо обусловленной функции;
            \item нарисовать графики с линиями уровня и траекториямиметодов для каждого случая;
        \end{enumerate}
    
    \item Реализовать генератор случайных квадратичных функций $n$ переменных с числом обусловленности $k$.
    \item
    Исследовать зависимость числа итераций $T(n, k)$, необходимых градиентному
    спуску для сходимости в зависимости от размерности пространства $2 \leq n \leq 10^3$
    и числа обусловленности оптимизируемой функции $1 \leq k \leq 10^3$
    \item Проведение множественных экспериментов и для  полученния среднего  значения числа итераций.
    \item Реализовать одномерный поиск с учетом условий Вольфе и исследовать его эффективность, а также сравнить полученные результаты с реализованными ранее методами.
\end{enumerate}


\endinput